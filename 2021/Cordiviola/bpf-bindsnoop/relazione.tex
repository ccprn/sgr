\documentclass[12pt]{report}
\usepackage{geometry}
\geometry{
	a4paper,
	left=15mm,
	right=15mm,
	top=15mm,
	bottom=15mm
	}
\usepackage[export]{adjustbox}
\usepackage{graphicx}
\usepackage[utf8]{inputenc}
\graphicspath{ {./img/} }

\begin{document}

\title{bpf-bindsnoop.py}
\author{Christian Cordiviola}
\maketitle
\section*{Introduction to eBPF}
eBPF, as Extended Berkeley Packet Filter, is a kernel technology (for Linux 4.x and later) that can run sandboxed programs in an OS kernel. The OS guarantees safety and execution efficiency as if natively compiled, using a JIT (Just-in-Time) compiler and a verification engine.
	Usage of eBPF today includes:
	\begin{itemize}
		\item Security
		\item Tracing
		\item Networking
		\item Kernel Observability \& Monitoring
	\end{itemize}
eBPF allows regular userspace applications to package logic to be executed within the Linux kernel as bytecode. These are called eBPF programs, written in C, and are produced by eBPF compiler toolchain called BCC (BPF Compiler Collection).\newline However, before being loaded into the kernel, an eBPF program must pass a set of checks. The verification engine executes the eBPF program within a virtual machine, allowing the verifier to traverse the potential execution paths the program may take when executed within the kernel, making sure it does run to completion without any loops, which would cause kernel lockup. Only if all checks pass, the eBPF program is loaded and compiled into the kernel and starts waiting for the right hook: once triggered, the bytecode executes.
\section*{Using eBPF}
eBPF programs execute in an event-driven environment: they are triggered by kernel hooks. Hook locations include:
	\begin{itemize}
		\item System Calls
		\item Function entry/exit
		\item Network Events
		\item Kprobes and uprobes
	\end{itemize}
When eBPF programs are triggered at their hook points, they can call helper functions that can perform a wide variety of tasks. Helpers, however, must be defined by the kernel, meaning there is a whitelist of calls eBPF programs can make. Additionally, eBPF programs make use of eBPF maps to keep state between invocations and to share data with user-space applications. The kernel expects all eBPF programs to be loaded as bytecode, which is created by using higher-level languages. The most popular toolchain for writing and debugging eBPF programs is BCC, based on LLVM and CLANG.	
\\[2\baselineskip]
\includegraphics{eBPF-architecture.jpg}
\\[2\baselineskip]
After verification, the bytecode is JIT (Just-in-Time) compiled into native machine code. eBPF is 64-bit encoded with 11 total registers, allowing it to map the hardware for x86\_64, ARM and arm64 architecture among others. The important takeaway is that eBPF unlocks access to kernel level events, without the typical restrictions found when changing kernel code directly.
\\[2\baselineskip]
Summarizing, eBPF works by:
	\begin{itemize}
		\item Compiling eBPF programs into bytecode
		\item Verifying programs execute safely in a VM before being loaded at the hook point
		\item Attaching programs to hook points within the kernel that are triggered by specific events
		\item Compiling at runtime for maximum efficiency
		\item Calling helper functions to manipulate dat when a program is triggered
		\item Using maps (key-value pairs) to share data between user and kernel space
	\end{itemize}
\clearpage
\section*{bindsnoop.py}
The tool utilizes eBPF to trace all IPv4 and IPv6 bind() attempts, even if they fail or the socket is not willing to accept(), by attaching an eBPF program to the kernel hooks \textbf{inet\_bind()} and \textbf{inet6\_bind}. This is known as attaching a kprobe (kernel probe) to the hooks.
\\[\baselineskip]
Syntax: \textit{kprobe\_\_kernel\_function\_name}
\\[\baselineskip]
\textit{kprobe\_\_} is a special prefix that creates a kprobe (dynamic tracing of a kernel function call) for the kernel function name provided as the remainder. kprobes can also be used by declaring a normal C function, then using the Python \textit{BPF.attach\_kprobe()} to associate it with a kernel function. This is done by using python3-bpfcc, the Python wrapper package for BCC.
\newline 
Arguments are specified on the function declaration:
\\[\baselineskip]
\textit{kprobe\_\_kernel\_function\_name(struct pt\_regs *ctx [, argument1 ...])}
\\[\baselineskip]
The first argument is always struct pt\_regs *, the remainder are the arguments to the function (they don't need to be specified, if you don't intend to use them).
\\[\baselineskip]
\includegraphics{attach_kprobes.jpg}
\\[\baselineskip]
The eBPF program gathers data, such as:
	\begin{itemize}
		\item pid of binding process
		\item IPv4/v6 bound address
		\item bound port
		\item protocol (TCP/UDP only)
	\end{itemize}
from the kernel call's arguments, the struct sock* and the struct sockaddr*, and fills the data\_t structure with said data, then sends them to user space by writing the structure in an eBPF map.
\\[\baselineskip]
\includegraphics{data_t.jpg}
\clearpage
The handle:
\\[2\baselineskip]
\includegraphics{the_kprobe.jpg}
\\[2\baselineskip]
This function is called whenever one of the hookpoints, \textbf{inet\_bind()} or \textbf{inet6\_bind()}, are triggered. Fills the data\_t structure with the necessary information and submits it to the map, so it can be accessed from user-space.
\clearpage
The user-space side of the program, written in Python, initializes eBPF, then polls the map waiting for a new bind() event. Whenever said event happens, the data\_t structure is retrieved, parsed and the output is built.
\\[2\baselineskip]
\includegraphics{user_side.jpg}
\end{document}

